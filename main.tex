% CQMUletter_example.tex - an example latex file to illustrate HITletter.cls
%
% Template by Brian Wood (brian.wood@oregonstate.edu).  Please feel free to send suggestions for changes; this template/cls is not exactly elegantly done!
% Modified by Jiaqiang Wang (jqcn.link@gmail.com) to fit the need of CQMU students, 
% based on the HIT version modified by MincooLee (mincoolee@gmail.com) which is coming from NJU version modified by Luyi Li (owenliluyi@gmail.com).

\documentclass[12pt]{CQMUletter}
\usepackage{fontspec} 
\usepackage{tikz} 
\usepackage{xcolor}
	\definecolor{CQMUgreen}{RGB}{23,98,102}
\usepackage{lipsum}
\usepackage{fancyhdr}
\usepackage{lastpage}
\usepackage{eso-pic}
\usepackage[base]{babel}
\usepackage[hidelinks]{hyperref}
%
% This section is just a bunch of busywork so that the second and following pages read ``Page X of Y''
\pagestyle{fancy}
\fancyhf{}
\renewcommand{\headrulewidth}{0pt}
\renewcommand{\footrulewidth}{0pt}
\rhead{Page \thepage\ of \pageref{LastPage}}
%
%
% Set custom font here. Comment this line out if you do not have a Cambria font (originally included with this template) installed; computer modern (or whatever your current default font is) will be substituted.
%
% \setmainfont{Cambria.ttf}[Path = font/, BoldFont  = {CambriaBold.ttf}, ItalicFont  = {CambriaItalic.ttf}, BoldItalicFont = {CambriaBoldItalic.ttf}]
\setmainfont{Times New Roman}

% The material below is a whole big dang thing whose purpose is just to set up a fixed coordinate system for \tikz so that you can put the Department or School address in the upper right-hand side without it moving all around every time you change something in the page.  I think it works.
% Defining a new coordinate system for the page:
%
% --------------------------
% |(-1,1)    (0,1)    (1,1)|
% |                        |
% |(-1,0)    (0,0)    (1,0)|
% |                        |
% |(-1,-1)   (0,-1)  (1,-1)|
% --------------------------
\makeatletter
\def\parsecomma#1,#2\endparsecomma{\def\page@x{#1}\def\page@y{#2}}
\tikzdeclarecoordinatesystem{page}{
    \parsecomma#1\endparsecomma
    \pgfpointanchor{current page}{north east}
    % Save the upper right corner
    \pgf@xc=\pgf@x%
    \pgf@yc=\pgf@y%
    % save the lower left corner
    \pgfpointanchor{current page}{south west}
    \pgf@xb=\pgf@x%
    \pgf@yb=\pgf@y%
    % Transform to the correct placement
    \pgfmathparse{(\pgf@xc-\pgf@xb)/2.*\page@x+(\pgf@xc+\pgf@xb)/2.}
    \expandafter\pgf@x\expandafter=\pgfmathresult pt
    \pgfmathparse{(\pgf@yc-\pgf@yb)/2.*\page@y+(\pgf@yc+\pgf@yb)/2.}
    \expandafter\pgf@y\expandafter=\pgfmathresult pt
}
\makeatother
%
%
%%%%%%%%%%% Put Personal Information Here %%%%%%%%%%%
%
\def\name{Name,\\
Position,\\
College of Medical Informatics, \\
Chongqing Medical University
}
%
% List your degree(s), licences, etc. here if you like.
%\def\What{, Your degrees, etc.} 
%
% Set the name of your Department or School here
% I honestly don't know why the negative spacing is necessary to get the alignment of the first line correct.  This must be a ``\tikz'' thing.
%%%%%%%%%%%%%%%%%%  School or Department %%%%%%%%%%%%%%%
\def\Where{\hspace{-1.2mm}\textbf{\color{CQMUgreen}
College of Medical Informatics, \\Chongqing Medical University
}} 
%%%%%%%%%%%%  Additional Contact Information %%%%%%%%%%%
%
% Set your preferred primary contact address here.
\def\Address{No. 61, University Town Middle Road, Shapingba District, Chongqing} 
%
\def\CityZip{Chongqing, 401334\\
P.R. China} 
%
% Set your e-mail here
\def\Email{\textbf{\color{CQMUgreen}E-mail}: \href{mailto:xxx@stu.cqmu.edu.cn}{xxx@stu.cqmu.edu.cn}}
%
% Set your preferred contact number here
\def\TEL{\textbf{\color{CQMUgreen}Phone}: +86 xxx xxxx xxxx}
%
% Set your department or personal website here
\def\Homepage{\textbf{\color{CQMUgreen}Homepage}: \href{https://www.cqmu.edu.cn/}{https://www.cqmu.edu.cn/}}
%
%%%%%%%%%%%%%%%%  Footer information  %%%%%%%%%%%%%%%%%%
%
%  The next line is for your college, used as a footer.  If you prefer not to have this, just comment out these lines in favor of the line labeled "[[Alternate]]" below
\def\school{\small{
  CQMU $\cdot$
     ~College of Medical Informatics $\cdot$
     ~No. 61, University Town Middle Road, Shapingba District, Chongqing $\cdot$
     ~Southwest, P.R. China} } 
% \def\school{~}  % [[Alternate]]
%
%%%%%%%%%%%%%%%%%%%%%  Signature line  %%%%%%%%%%%%%%%%%%%%%
%
% Set your signature line here.
% One can add a signature image in a PDF file using the following code; this requires a file called "signature_block.pdf" to be installed in the same folder as the .tex file.  The vertical spacing (\vspace) and the scaling will have to be adjusted to get things to look correct for your particular signature image. Alternatively, comment out the following line in favor of the one labeled "[[Alternate]]" if you want to sign a paper copy of the letter.
%

\signature{ 
\vspace{-12mm}\includegraphics[scale=0.40]{img/signature_block.pdf}\\\vspace{-2mm}
\name}

%\signature{\name}  % [[Alternate]]

% This block sets up the address on the right-hand side of the header. 
%
% The following lines just compile the information you set up into the LaTex letter variable "address" for later use.
%
%The following command "clears out" the default address so that it can be better set using \tikz
\address{}

\def\newaddress{
\Where\\ 
\Address\\ 
\CityZip\\ 
\TEL\\ 
\Email\\ 
\Homepage 
}
%
%%%%%%%%%%%  DATE  %%%%%%%%%%%%%%%%%%%%%%%%%
% If you want a date different from the current date, comment out the next line in favor of the line labeled "[[Alternate]]".  The ``\vspace{10mm}'' just moves the date down a tiny bit so it doesn't interfere with the header.  This can be adjusted to your preference.
%
\date{\vspace{10mm} \today}
%\date{\vspace{10mm} 20 September 2020}  %[[Alternate]]
%
%%%%%%%%%%% Set the subject here if there is one  %%%%
%\subject{Stuff} % optional subject line

\begin{document}
%
%
%%%%%%%%  The "To" address goes here.
%
\begin{letter}{
               Some University\\ 
               Some Addresss\\ 
               SomeTown, SomeState 					       				  		 ~~SomeZip
               }
% This line sets up the return address to the right-side of the OSU logo.  The location is set with absolute node addresses using ``\tikz''.  It can still be a bit fussy, and you may need to alter this a little to get things to look right.  The bit that changes the position are the numbers in parentheses ``at (14.2,2.7)''
%
\begin{tikzpicture}[remember picture,overlay,,every node/.style={anchor=center}]
\node[text width=7.5cm] at (page cs:0.45,0.73){\fontsize{11pt}{\baselineskip}\selectfont \newaddress};
\end{tikzpicture} 

%%%%%%  The ``opening'' is just the method of address you would like to use at the start of the letter.
%
\opening{Dear Application Committee,}

%%%%%%%%%% Body of letter   %%%%%%%%%%%%%%
% Remove it if you do not want watermark
\watermark{}{}{}

I hope this letter finds you well. I am writing to express my sincere interest in gaining admission to [University Name] for the upcoming academic year. As a highly motivated and academically driven individual, I am eager to contribute to the vibrant academic community at your esteemed institution.

Allow me to introduce myself. My name is [Your Name], and I am currently a senior student at [Current School/College/University], majoring in [Your Major]. Throughout my academic journey, I have consistently demonstrated a strong passion for learning and a commitment to excellence in all aspects of my studies.

I am particularly drawn to [University Name] because of its renowned faculty, diverse student body, and comprehensive range of academic programs. I am impressed by the university's dedication to fostering intellectual curiosity, critical thinking, and innovation, which align perfectly with my own educational goals and aspirations.

In addition to my academic pursuits, I have actively engaged in extracurricular activities that have helped me develop valuable leadership, communication, and teamwork skills. For instance, I have served as [Leadership Position] of [Club/Organization Name], where I have had the opportunity to organize events, collaborate with peers, and make a positive impact on campus.

Furthermore, I am eager to contribute to the university community through my involvement in [Specific Activity, Club, or Program at the University]. I am particularly interested in [mention any specific research, program, or initiative at the university that excites you and aligns with your interests].

I am confident that my academic achievements, combined with my extracurricular involvement and genuine enthusiasm for learning, make me a strong candidate for admission to [University Name]. I am fully committed to making the most of the opportunities available at your institution and contributing to its academic and cultural diversity.

Thank you for considering my application. I look forward to the possibility of joining the [University Name] family and embarking on this exciting new chapter of my academic journey. Please do not hesitate to contact me if you require any further information or documentation.
%%%%%%% ``closing'' sets the sign-off line.
\closing{Sincerely,}

% Comment out/in the lines below as necessary
%\encl{If an enclosure is provided, let them know what it is.}

%\ps{A postscript if that is a thing you do.}

%\cc{Someone Who Cares (and is copied).}

\end{letter}

\end{document}